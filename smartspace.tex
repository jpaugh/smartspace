% \smartspace: Smartly insert whitespace after a control sequence.
%
% Narrator: It is tempting to abbreviate a frequently use phrase or
% command to a single control sequence:
%
%   \def\TeXie{\TeX{}ophiliac}
%   \def\ignorance{\TeX{}ophobia}
%
%   My psychologist said there's no way to cure a \TeXie~---
%   but neither is there a cure for \ignorance!
%
% But to do this liberally can introduce whitespace errors, where the
% following whitespace is missing:
%
%   \TeXie{}s unite!
%   % Hang on, lemme fix the kerning...
%
% These bugs are subtle in the source~--- but not in the output! And the
% more heavily you use such abbreviations, the easier it is to fudge.
% This shouldn't be: abbreviations should make life easier, not
% harder...
%
% [Enter \smartspace, stage left]
%
%   \def\TeXie{\Tex{}ohpiliac\smartspace}
%
%   A \TeXie can hack it.
%
% (Narrator continues) Now, `\TeXies' gobbles up the space, as usual:
% but then \smartspace adds it back
%
% Inner voice: But wait! I've already paid the price: my .tex files are
% littered with `\cs\ ' already! Won't that give me /two/ spaces?
%
% Narrator: No, no it won't. \smartspace checks the next token to decide
% whether to inject a space or not: if it finds  `\ ', `\space', or
% punctuation, it doesn't inject an (extra) space.
%
% Inner voice: Okay, sure. But what if I don't want a space in other
% situations? Like when I say
%
%   \TeXie{}s unite!
%
% Narrator: In such special cases, you can suffix the macro with
% `\nospace', which tells \smartspace to do nothing, even against its
% better judgement: you da' boss!
%
%   \TeXie\nospace{}s unite!
%
% Inner voice (saucily): That's just awkward!
%
% Narrator (obviously miffed): Define another macro!
%
%   \def\TeXie{\Tex{}ohpiliac\smartspace}
%   \def\TeXies{\TeXie\nospace{}s}
%
% Better still, define a generic abbreviation macro, if you please!
%
%   \def\abbrev#1#2{%
%       \expandafter\def\csname #1\expandafter{#2}
%   }
%   \abbrev{TeXie}{\Tex{}ohpiliac\smartspace}
%   \abbrev{TeXies}{\TeXie\nospace{}s}
%
%   \TeXies unite!
%
% Inner voice (false bravado): Ayhha!, I couldda done that!
%
% Epilogue: Sometimes, \smartspace chokes, with one of several
% quite cryptic TeX errors. But the problem usually comes down to one
% thing:
%
%   Since \smartspace checks the next token, it must be preceded by at
%   least one token:
%
%       - \smartspace cannot appear at the end of input; remember to put
%       `\bye' (or even `\nospace' or `\relax') at the end.
%
%       - \emph{\TeXie} fails! Use \emph{\TeXie\nospace} instead
%
% After reading that list, you may think twice about using \smartspace,
% but consider this: when \smartspace chokes, you get an error message
% that~--- after a little (initial) confusion~--- can readily be fixed;
% when you choke, you get a document full of typos.
%
% In a longer document, one has many choices as to when to use
% \smartspace, and some judicious use can alleviate most problems.
% In particular, \smartspace is not needed with any macro that takes its
% arguments in braces, since TeX itself can then guess the spacing
% correctly. So, if you've got the following definitions for your new
% book~--- or Master's theses~--- don't change them:
%
%   \def\term#1{\emph{#1}}%     Term that I explain in my book
%   \def\work#1{\cite{#1}}%     Somebody's work that I mention
%
% But, /do/ add \smartspace to the following:
%
%   \def\lp{\work{my-last-paper}\smartspace}
%   \def\oneHump{\term{One-Hump Camel}\smartspace}
%   ...
%   My \oneHump can beat your \term{Asparagus} any day!
%
% This way, you'll avoid all issues with \smartspace where TeX can
% handle spacing correctly, and you'll avoid all spacing issues where
% TeX can't.
%
% What is your thesis /about/, anyway?
{\endlinechar=-1
\catcode`\@11

\global\let\nospace=\empty
\long\gdef\smartspace#1{
    \def\tmp@space{\space #1}
    \def\tmp@nospace{#1}
    \let\next=\tmp@space
    \ifx\nospace#1
        \let\next=\tmp@nospace
    \else
        \ifx\ #1% Explicit space
            \let\next=\tmp@nospace
        \else
            \ifx\space#1% Explicit space
                \let\next=\tmp@nospace
            \else
                \ifcat\noexpand#1\relax
                    % Other control sequences
                \else
                    \ifcat;#1% Punctuation
                        \ifx(#1
                        \else
                            \let\next=\tmp@nospace
                        \fi
                    \fi
                \fi
            \fi
        \fi
    \fi
    \next
}
}%\endlinechar
